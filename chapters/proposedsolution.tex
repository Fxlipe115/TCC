\section{Libraries and Tools Used}

The following libraries and tools were used in the implementation:

\begin{itemize}
    \item \textbf{Charm-Crypto}: A Python library for pairing-based cryptography, used to implement the attribute-based encryption scheme.
    \item \textbf{Flask}: A lightweight web framework for building the API endpoints for encryption and decryption.
    \item \textbf{Flask-RESTX}: An extension for Flask that adds support for creating REST APIs.
    \item \textbf{PyCryptodome}: A self-contained Python package of low-level cryptographic primitives, used for AES encryption and decryption.
    \item \textbf{HashiCorp Vault}: A tool for securely storing and accessing secrets, used to manage cryptographic keys.
\end{itemize}




\section{Process of Creating the AES Key}

The process of creating the AES key involves the generation of a symmetric key using the attribute-based encryption scheme and hashing the key to the appropriate length for AES encryption.

\begin{enumerate}
    \item \textbf{Generate Symmetric Key}: A symmetric key is generated using the attribute-based encryption scheme. This key is derived based on the specified access policy and the public keys of the authorities.
    \item \textbf{Serialize Symmetric Key}: The symmetric key is serialized to a hexadecimal string for further processing.
    \item \textbf{Hash Symmetric Key}: The serialized symmetric key is hashed using the SHA-256 hash function to produce a 256-bit key suitable for AES encryption.
\end{enumerate}

Mathematically, the process can be described as follows:

\begin{equation}
\text{hashed\_key} = \text{SHA-256}(\text{serialized\_symmetric\_key})
\end{equation}
\subsection{AES Encryption and Decryption}

The AES encryption and decryption process involves using the hashed symmetric key to encrypt and decrypt the payload data. The steps are as follows:

\subsubsection{Encryption}

\begin{enumerate}
  \item \textbf{Initialize Pairing Group}: Initialize the pairing group \(G_T\).
  \item \textbf{Generate Random Element}: Generate a random element \(gt\) in \(G_T\).
  \item \textbf{Retrieve Public Keys}: Retrieve the public keys from the key manager for the specified authorities.
  \item \textbf{Encrypt with MA-ABE}: Use the MA-ABE scheme to encrypt the random element \(gt\) with the public keys and access policy.
  \item \textbf{Serialize Symmetric Key}: Serialize the symmetric key to a hexadecimal string.
  \item \textbf{Hash Symmetric Key}: Hash the serialized symmetric key using SHA-256 to produce a 256-bit key.
  \item \textbf{Initialize AES Cipher}: Initialize an AES cipher object in CBC mode using the hashed symmetric key.
  \item \textbf{Pad Payload}: Pad the payload data to ensure it is a multiple of the AES block size.
  \item \textbf{Encrypt Payload}: Encrypt the padded payload using the AES cipher.
\end{enumerate}

Pseudo-code for the encryption process:

\begin{align}
  \text{initialize\_pairing\_group}(G_T) \\
  gt = \text{generate\_random\_element}(G_T) \\
  \text{public\_keys} = \text{retrieve\_public\_keys}(\text{authorities}) \\
  \text{symmetric\_key} = \text{maabe\_encrypt}(\text{public\_keys}, gt, \text{policy}) \\
  \text{serialized\_symmetric\_key} = \text{serialize}(\text{symmetric\_key}) \\
  \text{hashed\_key} = \text{SHA-256}(\text{serialized\_symmetric\_key}) \\
  \text{aes\_cipher} = \text{AES.new}(\text{hashed\_key}, \text{AES.MODE\_CBC}) \\
  \text{ciphertext} = \text{aes\_cipher.encrypt}(\text{pad}(\text{payload}))
  \end{align}

\subsubsection{Decryption}

\begin{enumerate}
    \item \textbf{Initialize AES Cipher}: An AES cipher object is initialized in CBC mode using the hashed symmetric key.
    \item \textbf{Decrypt Ciphertext}: The ciphertext is decrypted using the AES cipher.
    \item \textbf{Unpad Decrypted Data}: The decrypted data is unpadded to retrieve the original payload.
\end{enumerate}

Pseudo-code for the decryption process:

\begin{verbatim}
hashed_key = SHA-256(serialized_symmetric_key)
aes_cipher = AES.new(hashed_key, AES.MODE_CBC)
decrypted_payload = unpad(aes_cipher.decrypt(ciphertext))
\end{verbatim}