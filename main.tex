\documentclass[cic,tc,english]{iiufrgs}

\usepackage[utf8]{inputenc}
\usepackage{graphicx}
\usepackage{times}
% \usepackage{palatino}
% \usepackage{mathptmx}       % p/ usar fonte Adobe Times nas fórmulas

\usepackage[alf,abnt-emphasize=bf]{abntex2cite}

\usepackage{lipsum}

\usepackage{revisornotes}

\usepackage{draftenvironment}

\usepackage{ragged2e}

\usepackage{listings}

\usepackage{amsmath}      % For math symbols
\usepackage[boxed]{algorithm}    % For the algorithm environment
\usepackage{algpseudocode} % For pseudocode formatting
\usepackage{xcolor}       % Optional for colored comments

\usepackage{tikz}

\title{
    A Hybrid Multi-Authority Attribute-Based Encryption scheme implementation for securing hIoT data
}
\translatedtitle{
    Esquema Híbrido de Criptografia Baseada em Atributos de Múltiplas Autoridades para Transmissão Segura de Dados hIoT
}

\author{Graeff}{Felipe de Almeida}

\advisor[Prof.~Dr.]{Nobre}{Jéferson Campos}

% \date{maio}{2001}
% \location{Itaquaquecetuba}{SP}

\keyword{cryptographic algorithms}
\keyword{IPFS}
\keyword{decentralized technologies}
\keyword{performance analysis}
\keyword{security evaluation}
\keyword{data storage}
\keyword{data exchange}

\translatedkeyword{algoritmos criptográficos}
\translatedkeyword{IPFS}
\translatedkeyword{tecnologias descentralizadas}
\translatedkeyword{análise de desempenho}
\translatedkeyword{avaliação de segurança}
\translatedkeyword{armazenamento de dados}
\translatedkeyword{intercâmbio de dados}

%\settowidth{\seclen}{1.10~}

\begin{document}

\maketitle

% dedicatory
\clearpage
\begin{flushright}
    \mbox{}\vfill
    \begin{tabular}{p{0.55\linewidth} p{0.45\linewidth}}
        In loving memory of Professor Raul Fernando Weber, a remarkable educator and influential figure in the field of computer security. His passion for teaching and dedication to his students left an indelible mark on all who had the privilege to learn from him. As I embark on my undergraduate thesis in computer security, I humbly dedicate this work to the memory of a great professor and a charismatic person who inspired generations of students, including myself. His guidance and enthusiasm ignited my love for this area, and I am forever grateful for the knowledge and inspiration he shared. His legacy lives on in the pursuit of knowledge and excellence. Rest in peace, \mbox{Professor} Weber.\\
    \end{tabular}
\end{flushright}

\chapter*{Acknowledgements}
    I would like to express my deepest gratitude to my advisor, Professor 
    Jéferson Campos Nobre, for his invaluable guidance and expertise throughout 
    my research journey. His patience and profound knowledge have not only 
    shaped this work but have profoundly influenced my development in the field.

    I am immensely thankful to Laura, whose foundational research was 
    instrumental in this thesis. Her dedication and innovative approach have 
    significantly contributed to the success of this project.

    My appreciation also extends to the faculty and staff at the Informatics 
    Department. The education and support provided by the professors have
    been fundamental to my academic and professional growth. Their dedication to
    creating a nurturing and challenging academic environment has been critical 
    to my development.

    Special thanks go to Valéria, my best friend, for her constant support and 
    love. Her encouragement has been a pillar of strength in difficult times, 
    reminding me of the value of perseverance.

    I am grateful to all my friends for their continued encouragement and belief
    in my abilities. Their support and the occasional much-needed distractions 
    have helped me maintain my focus and passion for my research.

    Finally, I would like to thank Marina and Pedro, my psychologist and 
    psychiatrist. Their professional support has been crucial in managing the 
    stress of my academic commitments, significantly contributing to my 
    well-being and success.

    This thesis reflects the collective support and encouragement of everyone 
    mentioned, and I am deeply thankful to each of you.

\clearpage
\begin{flushright}
    \mbox{}\vfill
    \begin{minipage}{0.49\textwidth}
        % \setlength{\baselineskip}{1.2em} % Adjust line spacing
        \justifying
        \noindent
        {\itshape
        ``That is what I have always understood to
        be the essence of anarchism: the conviction
        that the burden of proof has to be placed on 
        authority, and that it should be dismantled 
        if that burden cannot be met.''\\
        }
        
        \raggedleft
        \parbox{\linewidth}{--- \textsc{Noam Chomsky,} \textit{Red and Black Revolution, No. 2} (May 1995)}
    \end{minipage}
\end{flushright}


\begin{abstract}
    \lipsum[1]
\end{abstract}

\begin{translatedabstract}
    \input{chapters/translatedabstract.tex}
\end{translatedabstract}

\listoffigures

\listoftables

\listofalgorithms

\begin{listofabbrv}{MA-ABE} % longest abbreviation
    \item[AES] Advanced Encryption Standard
    \item[DLP] Discrete Logarithm Problem
    \item[ECC] Elliptic Curve Cryptography
    \item[EHR] Electronic Health Record
    \item[EMR] Electronic Medical Record
    \item[HIoT] Health Internet of Things
    \item[IoMT] Internet of Medical Things
    \item[IPFS] InterPlanetary File System
    \item[MA-ABE] Multi-Authority Attribute-Based Encryption
    \item[PoS] Proof of Stake
    \item[PoW] Proof of Work
    \item[RSA] Rivest-Shamir-Adleman
    \item[SHA] Secure Hash Algorithm
    \item[SIFF] Secure Identity Federation Framework
\end{listofabbrv}

\begin{listofsymbols}{$\alpha\beta\pi\omega$}
    \item[$E_{K_{G_T}}$] Encrypted element $K_{G_T}$
    \item[$E_{\text{payload}}$] Encrypted payload data
    \item[$K_A$] Authorities' public keys
    \item[$K_{G_T}$] Element in the pairing group $G_T$
    \item[$K_{\text{SHA}}$] Hashed symmetric key
    \item[$K_{\text{user}}$] User attribute keys
\end{listofsymbols}

\tableofcontents

\chapter{Introduction}
    % contextualização 1
\begin{draft}{Talk about lauras paper}
    Laura's paper ...
    \ToDo{Perguntar pra Laura em que camada do framework dela entra a criptografia}    
\end{draft}


% contextualização 2 - mais especificacao
\begin{draft}{Talk specifities about the technologies used in the framework and why security is relevant and not adressed by IPFS and Hyperledger}
    IPFS \cite{benet2013ipfs} is a peer-to-peer (P2P) decentralized storage system based on a content-addressable block storage model.
\end{draft}


% trabalhos relacionacios
\begin{draft}{Talk about related works}
    
\end{draft}

\begin{draft}{Gap}
    No entanto, nenhum dos trabalhos ...
\end{draft}

% proposta

In the present thesis we aim to propose a cryptography framework that best fits the solution introduced by \citet{laura2023}.

% (opcional) avaliação (implementação, experimentos, resultados)

\begin{draft}{Quick introduction of each chapter}
    In Section \ref{relatedworks},...

\end{draft}


\chapter{Background}
    \label{background}
    \citet{laura2023} proposed a new framework for securing Health Internet of Things (HIoT) data storage using Hyperledger and InterPlanetary File System (IPFS) \cite{benet2013ipfs}.

\section{Related works}
\label{relatedworks}

\Laura{pela quantidade de texto acho que ficaria melhor fazer um capítulo de background primeiro com a parte de ECC, ABE etc e alguma coisa de blockchain, e aí colocar o reated depois em um capítulo separado}
\Laura{o problem statement daí pode vir no final do capítulo de related}

    This section delves into the application of blockchain technology in healthcare, exploring its impact on data security, privacy enhancement, operational efficiency, and additional technological perspectives. Each study discussed is highlighted for its contributions and relevance to advancing blockchain in healthcare.

    \ToDo{Explicar o método utilizado par chegar nos artigos}

    \ToDo{This chapter is divided as follows...}

    
    \ToDo{Descrever cada item antes de descrever os trabalhos}
    \subsection{Data Security} \citet{Memos2021} propose a layered cloud architecture incorporating Advanced Encryption Standard (AES) and Rivest-Shamir-Adleman (RSA) encryption to enhance the security of e-health data transmission. \citet{Tian2019} use Secure Identity Federation Framework (SIFF) and Hyperledger Fabric to ensure medical data's integrity, availability, and privacy. \citet{Liu2024} develop a blockchain-based Electronic Medical Record (EMR) sharing scheme with IPFS for secure storage and proxy re-encryption for access control. Additionally, \citet{Esposito2018} investigate the use of blockchain to enhance data security in cloud-based healthcare systems, addressing vulnerabilities inherent in traditional security models.
    
    \subsection{Privacy Enhancement} \citet{Dwivedi2019} propose a blockchain framework for IoT devices in healthcare that minimizes computational demands while enhancing privacy. \citet{Esfahani2024} introduce a comprehensive privacy scheme using zero-knowledge proofs and ring signatures for IoT-based systems. \citet{Zala2024} enhance e-health data privacy through an attribute-based encryption scheme that improves anonymity. \citet{Li2024} create a regulated medical data trading scheme that ensures privacy and compliance using blockchain and zero-knowledge proofs.
    
    \subsection{Operational Efficiency} \citet{Vazirani2019} review the effectiveness of blockchain in managing healthcare records, noting improvements in interoperability. \citet{Shen2019} introduce MedChain, a framework that efficiently manages and shares healthcare data. \citet{Bhansali2022} integrate ciphertext-policy attribute-based encryption with federated learning for data management in the IoMT. Furthermore, \citet{XuJie2019} develop 'Healthchain,' a scalable blockchain-based framework for privacy-preserving health data management.
    
    \subsection{Cross-Border Data Sharing and Scalability} \citet{Rahman2020} describe a blockchain-based platform for secure cross-border data sharing, enhancing trust and security. \citet{Saeed2022} examine the potential of blockchain in healthcare, particularly its impact on managing, distributing, and processing medical records.
    
    \subsection{Additional Technological Perspectives} \citet{Hema2019} assess ECC-based encryption mechanisms for securing healthcare data in cloud environments. \citet{Naz2024} explore defences against quantum computing threats in blockchain through exotic signatures. \citet{Eghmazi2024} focus on securing IoT data using Hyperledger Fabric for scalable solutions. \citet{XuChang2019} discuss secure data sharing in cloud-assisted healthcare systems using cryptographic techniques. Lastly, \citet{Karaca2019} explores mobile cloud computing applications in stroke care, indirectly underscoring the need for integrating secure and efficient technology solutions in healthcare.

    \ToDo{Passar parágrafo abaixo para próxima sessão?}
    This comprehensive review highlights the varied applications of blockchain technology and the significant potential in enhancing healthcare systems. Although the studies discussed provide a solid foundation for advancing blockchain applications in healthcare, they also highlight several research gaps that require further investigation.

\section{Problem Statement}
    \Graeff{Não tenho certeza se é melhor colocar essa discussão dos gaps da literatura aqui como sessão. Aceito sugestões.}

    This section outlines research areas in blockchain technology for healthcare that require further exploration. These areas, identified based on the limitations and gaps in current literature, present significant opportunities to advance the field. This thesis aims to address these opportunities through targeted research objectives.

    \subsection{Scalability and Real-World Applicability} 
        A critical gap in current research on blockchain in healthcare is the need for substantial empirical evidence regarding the scalability of the proposed frameworks and real-world applicability. There is an urgent need to conduct large-scale pilot studies and real-world implementations to substantiate the theoretical advantages of blockchain in this sector. Such research is vital to validate the benefits and to pinpoint potential challenges in diverse healthcare settings. Addressing these gaps is essential for generating concrete data and insights that can guide the scalable deployment of blockchain technologies in practical environments.

    \subsection{Regulatory and Ethical Considerations} 
        Understanding its regulatory and ethical implications is crucial as blockchain technology becomes more integrated into healthcare systems. There is a pressing need to investigate how to design blockchain solutions to comply with healthcare regulations and ethical standards. This exploration should focus on critical areas such as patient privacy, data security, and consent management, encompassing theoretical and practical applications. Ensuring blockchain solutions are legally compliant and ethically sound while maintaining operational flexibility and scalability is fundamental to successful integration into healthcare systems. Addressing these considerations is vital for building trust and accepting blockchain technology in healthcare.

    \subsection{Cybersecurity Challenges} 
        The advancement of blockchain technology in healthcare must continuously adapt to the evolving landscape of cybersecurity threats. Significant opportunities exist to develop sophisticated defence mechanisms specifically designed to address the unique security challenges of blockchain's decentralized nature. Critical areas for further exploration include advancing cryptographic techniques, enhancing consensus algorithms, and implementing robust access control frameworks. These elements protect sensitive health data against unauthorized access and cyber-attacks. Addressing these critical vulnerabilities is imperative to ensuring the security and privacy of healthcare information managed via blockchain. By fortifying these areas, we can significantly enhance the trustworthiness and reliability of blockchain applications in healthcare settings.

\section{Elliptic Curve and Pairing-Based Cryptography}

The key creation process in our system leverages elliptic curve cryptography (ECC) and pairing-based cryptography. These cryptographic techniques provide a high level of security and efficiency, making them suitable for attribute-based encryption.

\subsection{Elliptic Curve Cryptography}

Elliptic curve cryptography is based on the algebraic structure of elliptic curves over finite fields. An elliptic curve is defined by an equation of the form:

\begin{equation}
y^2 = x^3 + ax + b
\end{equation}

where \(a\) and \(b\) are constants that define the specific curve. The points on the elliptic curve, along with a special point at infinity, form an additive group.

\subsection{Pairing-Based Cryptography}

Pairing-based cryptography involves a bilinear map, known as a pairing, that takes two elements from two different groups and maps them to a third group. The pairing operation is denoted as:

\begin{equation}
e: G_1 \times G_2 \rightarrow G_T
\end{equation}

where \(G_1\) and \(G_2\) are additive groups of points on elliptic curves, and \(G_T\) is a multiplicative group. The pairing operation has the following properties:

\begin{itemize}
    \item \textbf{Bilinearity}: \(e(aP, bQ) = e(P, Q)^{ab}\) for all \(P \in G_1\), \(Q \in G_2\), and integers \(a, b\).
    \item \textbf{Non-degeneracy}: There exist points \(P \in G_1\) and \(Q \in G_2\) such that \(e(P, Q) \neq 1\).
    \item \textbf{Computability}: The pairing \(e(P, Q)\) can be efficiently computed.
\end{itemize}        
    % coisas de criptografia!!!
    % curvas elipticas etc
    

%\chapter{Related Works}
    %\section{Contexto Geral}
    %\section{Contexto Específico}

\chapter{Proposed Solution}
    \label{proposedsolution}
    \section{Libraries and Tools Used}

The following libraries and tools were used in the implementation:

\begin{itemize}
    \item \textbf{Charm-Crypto}: A Python library for pairing-based cryptography, used to implement the attribute-based encryption scheme.
    \item \textbf{Flask}: A lightweight web framework for building the API endpoints for encryption and decryption.
    \item \textbf{Flask-RESTX}: An extension for Flask that adds support for creating REST APIs.
    \item \textbf{PyCryptodome}: A self-contained Python package of low-level cryptographic primitives, used for AES encryption and decryption.
    \item \textbf{HashiCorp Vault}: A tool for securely storing and accessing secrets, used to manage cryptographic keys.
\end{itemize}




\section{Process of Creating the AES Key}

The process of creating the AES key involves the generation of a symmetric key using the attribute-based encryption scheme and hashing the key to the appropriate length for AES encryption.

\begin{enumerate}
    \item \textbf{Generate Symmetric Key}: A symmetric key is generated using the attribute-based encryption scheme. This key is derived based on the specified access policy and the public keys of the authorities.
    \item \textbf{Serialize Symmetric Key}: The symmetric key is serialized to a hexadecimal string for further processing.
    \item \textbf{Hash Symmetric Key}: The serialized symmetric key is hashed using the SHA-256 hash function to produce a 256-bit key suitable for AES encryption.
\end{enumerate}

Mathematically, the process can be described as follows:

\begin{equation}
\text{hashed\_key} = \text{SHA-256}(\text{serialized\_symmetric\_key})
\end{equation}
\subsection{AES Encryption and Decryption}

The AES encryption and decryption process involves using the hashed symmetric key to encrypt and decrypt the payload data. The steps are as follows:

\subsubsection{Encryption}

\begin{enumerate}
  \item \textbf{Initialize Pairing Group}: Initialize the pairing group \(G_T\).
  \item \textbf{Generate Random Element}: Generate a random element \(gt\) in \(G_T\).
  \item \textbf{Retrieve Public Keys}: Retrieve the public keys from the key manager for the specified authorities.
  \item \textbf{Encrypt with MA-ABE}: Use the MA-ABE scheme to encrypt the random element \(gt\) with the public keys and access policy.
  \item \textbf{Serialize Symmetric Key}: Serialize the symmetric key to a hexadecimal string.
  \item \textbf{Hash Symmetric Key}: Hash the serialized symmetric key using SHA-256 to produce a 256-bit key.
  \item \textbf{Initialize AES Cipher}: Initialize an AES cipher object in CBC mode using the hashed symmetric key.
  \item \textbf{Pad Payload}: Pad the payload data to ensure it is a multiple of the AES block size.
  \item \textbf{Encrypt Payload}: Encrypt the padded payload using the AES cipher.
\end{enumerate}

Pseudo-code for the encryption process:

\begin{align}
  \text{initialize\_pairing\_group}(G_T) \\
  gt = \text{generate\_random\_element}(G_T) \\
  \text{public\_keys} = \text{retrieve\_public\_keys}(\text{authorities}) \\
  \text{symmetric\_key} = \text{maabe\_encrypt}(\text{public\_keys}, gt, \text{policy}) \\
  \text{serialized\_symmetric\_key} = \text{serialize}(\text{symmetric\_key}) \\
  \text{hashed\_key} = \text{SHA-256}(\text{serialized\_symmetric\_key}) \\
  \text{aes\_cipher} = \text{AES.new}(\text{hashed\_key}, \text{AES.MODE\_CBC}) \\
  \text{ciphertext} = \text{aes\_cipher.encrypt}(\text{pad}(\text{payload}))
  \end{align}

\subsubsection{Decryption}

\begin{enumerate}
    \item \textbf{Initialize AES Cipher}: An AES cipher object is initialized in CBC mode using the hashed symmetric key.
    \item \textbf{Decrypt Ciphertext}: The ciphertext is decrypted using the AES cipher.
    \item \textbf{Unpad Decrypted Data}: The decrypted data is unpadded to retrieve the original payload.
\end{enumerate}

Pseudo-code for the decryption process:

\begin{verbatim}
hashed_key = SHA-256(serialized_symmetric_key)
aes_cipher = AES.new(hashed_key, AES.MODE_CBC)
decrypted_payload = unpad(aes_cipher.decrypt(ciphertext))
\end{verbatim}
    %\section{Resumo do Trabalho da Laura aqui}
    %\section{Modificações que o Grefo fez em cima}


\chapter{Experimental Setup}
% coisas técnicas de implementação etc
% biblotecas que vc usou etc
% código etc

% https://github.com/JHUISI/charm
% scheme abenc_maabe_rw15

\section{Machine Specifications}
The experiments were conducted on a workstation with the following specifications:

\begin{itemize}
    \item \textbf{Operating System:} Linux Mint 22 (x86\_64)
    \item \textbf{Host:} HP ZBook Firefly 14 inch G10 A Mobile Workstation PC SBKPF
    \item \textbf{Kernel:} 6.8.0-49-generic
    \item \textbf{Processor:} AMD Ryzen 5 PRO 7640HS with Radeon 760M Graphics (12 cores) @ 5.015GHz
    \item \textbf{Memory:} 12,599 MiB used out of 15,260 MiB total
\end{itemize}



\chapter{Evaluation}
    \label{evaluation}
    \section{Implementation}

\section{Environmental Settings}

\section{Results/Discussion}
% teste de performance
% tempo de execução (mpv)


\chapter{Conclusion}
    \input{chapters/conclusion.tex}

% \section{Figuras e tabelas}

% Esta seção faz referência às Figuras~\ref{fig:estrutura},~\ref{fig:ex1} e~\ref{fig:ex2}, a título de exemplo. A primeira figura apresenta a estrutura de uma figura. A \emph{descrição} deve aparecer \textbf{acima} da figura. Abaixo da figura, deve ser indicado a origem da imagem, mesmo se essa for apenas os autores do texto.

% A Figura~\ref{fig:ex1} representa o caso mais comum, onde a figura propriamente dita é importada de um arquivo ( neste exemplo o formato é \texttt{eps} ou \texttt{pdf}. Veja a seção \ref{sec:fig_format}). A Figura~\ref{fig:ex2} exemplifica o uso do environment \texttt{picture}, para desenhar usando o próprio~\LaTeX.

% \begin{figure}[h]
%     \caption{Descrição da Figura deve ir no topo}
%     \begin{center}
%         % Aqui vai um includegraphics , um picture environment ou qualquer
%         % outro comando necessário para incorporar o formato de imagem
%         % utilizado.       bs 
%         \begin{picture}(100,100)
%             \put(0,0){\line(0,1){100}}
%             \put(0,0){\line(1,0){100}}
%             \put(100,100){\line(0,-1){100}}
%             \put(100,100){\line(-1,0){100}}
%             \put(10,50){Uma Imagem}
%         \end{picture}    
%     \end{center}
%     \label{fig:estrutura}
%     \legend{Fonte: Os Autores}
% \end{figure}

% \begin{figure}
%     \caption{Exemplo de figura importada de um arquivo e também exemplo de caption muito grande que ocupa mais de uma linha na Lista~de~Figuras}
%     \begin{center}
%         \includegraphics[width=8em]{images/fig.jpg}
%     \end{center}
%     \legend{Fonte: Os Autores}
%     \label{fig:ex1}
% \end{figure}

% % o `[h]' abaixo é um parâmetro opcional que sugere que o LaTeX coloque a
% % figura exatamente neste ponto do texto. Somente preocupe-se com esse tipo
% % de formatação quando o texto estiver completamente pronto (uma frase a mais
% % pode fazer o LaTeX mudar completamente de idéia sobre onde colocar as
% % figuras e tabelas)
% % \begin{figure}[h]
% \begin{figure}
%     \caption{Exemplo de figura desenhada com o environment \texttt{picture}.}
%     \begin{center}
%         \setlength{\unitlength}{.1em}https://www.overleaf.com/project/62e9d968111a266d82bc89ed
%         \begin{picture}(100,100)
%             \put(20,20){\circle{20}}
%             \put(20,20){\small\makebox(0,0){a}}
%             \put(80,80){\circle{20}}
%             \put(80,80){\small\makebox(0,0){b}}
%             \put(28,28){\vector(1,1){44}}
%         \end{picture}
%     \end{center}
%     \legend{Fonte: Os Autores}
%     \label{fig:ex2}
% \end{figure}

% Tabelas são construídas com praticamente os mesmos comandos. Ver a tabela \ref{tbl:ex1}.

% \begin{table}[h]
%     \caption{Uma tabela de Exemplo}
%     % OBS: não use \begin{center}, pois este aumenta o espaçamento entre a caption/legend e a tabela
%     % Para figuras, a aparência é melhor com o espaçamento extra
%     \centering
%         \begin{tabular}{c|c|p{5cm}}
%           \hline
%           \textit{Col 1}  &   \textit{Col 2}  &   \textit{Col 3} \\
%           \hline
%           \hline
%           Val 1           &   Val 2           & Esta coluna funciona como um parágrafo, tendo uma margem definida em 5cm. Quebras de linha funcionam como em qualquer parágrafo do tex. \\
%           Valor Longo     & Val 2             & Val 3 \\
%           \hline
%         \end{tabular}
%     \legend{Fonte: Os Autores}
%     \label{tbl:ex1}
% \end{table}

% \subsection{Formato de Figuras}
% \label{sec:fig_format}

% O LaTeX permite utilizar vários formatos de figuras, entre eles \emph{eps}, \emph{pdf}, \emph{jpeg} e \emph{png}. Programas de diagramação como Inkscape (e mesmo LibreOffice) permitem gerar arquivos de imagens vetoriais que podem ser utilizados pelo LaTeX sem dificuldade. Pacotes externos permitem utilizar SVG e outros formatos.

% Dia e xfig são programas utilizados por dinossauros para gerar figuras vetoriais. Se possível, evite-os.

% \subsection{Classificação dos etc.}

% O formato do instituto de informática define 5 níveis: capítulo, seção, subseção e outros 2 sem nome.

% \subsubsection{Subsubseção}
% Exemplo de uma subsubseção.

% \paragraph{Parágrafo}
% Exemplo de um parágrafo.

% \section{Sobre as referências bibliográficas}

% A classe \emph{iiufrgs} faz uso do pacote \emph{abnTeX2} com algumas alterações
% feitas por Sandro Rama Fiorini. Culpe ele se algo der errado. Agradeça a ele
% pelo que der certo. As modificações dão uma camada de tinta NATBIB-style,
% já que o abntex2 usa uns comandos de citação feitos para alienígenas de 5 braços
% wtf. Exemplos de citação:

% \begin{itemize}
%     \item \emph{cite}: Unicórnios são verdes \cite{Adams2009Conceptual};
%     \item \emph{citep}:Unicórnios são verdes \citep{Adams2009Conceptual};
%     \item \emph{citet}: Segundo \citet{Adams2009Conceptual}, unicórnios são
%     verdes.
%     \item \emph{citen or citenum}: Segundo \citen{Adams2009Conceptual},
%     unicórnios são verdes.
%     \item \emph{citeauthor e citeyearpar}: Segundo artigos de
%     \citeauthor{Adams2009Conceptual} , unicórnios são verdes
%     \citeyearpar{Adams2009Conceptual}.

% \end{itemize}

% O estilo abnt fornecido antigamente pelo UTUG não é mais recomendado, pois não
% produz saída de acordo com as exigências da biblioteca.

% Recomenda-se o uso de bibtex para gerenciar as referências (veja o arquivo
% biblio.bib).

% \section{Mais uma Seção}

% Agora vamos fazer várias seções para termos valores de 2 dígitos no \contentsname.

% \section{Mais uma Seção}
% \section{Mais uma Seção}
% \section{Mais uma Seção}
% \section{Mais uma Seção}
% \section{Mais uma Seção}
% \section{Mais uma Seção}
% \section{Mais uma Seção}
% \section{Mais uma Seção}
% \section{Mais uma Seção}


% referências
% aqui será usado o environment padrao `thebibliography'; porém, sugere-se
% seriamente o uso de BibTeX e do estilo abnt.bst (veja na página do
% UTUG)
% 
% observe também o estilo meio estranho de alguns labels; isso é
% devido ao uso do pacote `natbib', que permite fazer citações de
% autores, ano, e diversas combinações desses

\bibliographystyle{abntex2-alf}
\bibliography{biblio}

\end{document}
