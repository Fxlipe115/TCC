\documentclass[cic,tc,english]{iiufrgs}

\usepackage[utf8]{inputenc}
\usepackage{graphicx}
\usepackage{times}
% \usepackage{palatino}
% \usepackage{mathptmx}       % p/ usar fonte Adobe Times nas fórmulas

\usepackage[alf,abnt-emphasize=bf]{abntex2cite}

\usepackage{lipsum}

\newcommand\new[1]{{\color{red}{\fbox{#1}}}}

\usepackage{tcolorbox}

\newcommand{\newRevisor}[3]{
  \expandafter\newcommand\csname #1\endcsname[1]{
    \begin{tcolorbox}[colback=#2]
      \textcolor{#3}{\bfseries #1: ##1}
    \end{tcolorbox}
  }
}

\newRevisor{Graeff}{Cyan}{black}
\newRevisor{Nobre}{orange}{black}
\newRevisor{ToDo}{red}{black}


\title{
    Securing Distributed Data with Cryptographic Algorithms: A Case Study on 
    IPFS
}
\translatedtitle{
    Assegurando Dados Distribuídos com Algoritmos Criptográficos: Um Estudo de 
    Caso no IPFS
}

\author{Graeff}{Felipe de Almeida}

\advisor[Prof.~Dr.]{Nobre}{Jéferson Campos}

% \date{maio}{2001}
% \location{Itaquaquecetuba}{SP}

\keyword{cryptographic algorithms}
\keyword{IPFS}
\keyword{decentralized technologies}
\keyword{performance analysis}
\keyword{security evaluation}
\keyword{data storage}
\keyword{data exchange}

\translatedkeyword{algoritmos criptográficos}
\translatedkeyword{IPFS}
\translatedkeyword{tecnologias descentralizadas}
\translatedkeyword{análise de desempenho}
\translatedkeyword{avaliação de segurança}
\translatedkeyword{armazenamento de dados}
\translatedkeyword{intercâmbio de dados}

%\settowidth{\seclen}{1.10~}

\begin{document}

\maketitle

% dedicatoria
\clearpage
\begin{flushright}
    \mbox{}\vfill
    \begin{tabular}{p{0.55\linewidth} p{0.45\linewidth}}
        In loving memory of Professor Raul Fernando Weber, a remarkable educator and influential figure in the field of computer security. His passion for teaching and dedication to his students left an indelible mark on all who had the privilege to learn from him. As I embark on my undergraduate thesis in computer security, I humbly dedicate this work to the memory of a great professor and a charismatic person who inspired generations of students, including myself. His guidance and
        ignited my love for this area, and I am forever grateful for the knowledge and inspiration he shared. His legacy lives on in the pursuit of knowledge and excellence. Rest in peace, Professor Weber.\\
    \end{tabular}
\end{flushright}

\chapter*{Acknowledgements}
    I would like to express my deepest gratitude to my advisor, Professor 
    Jéferson Campos Nobre, for his invaluable guidance and expertise throughout 
    my research journey. His patience and profound knowledge have not only 
    shaped this work but have profoundly influenced my development in the field.

    I am immensely thankful to Laura, whose foundational research was 
    instrumental for this thesis. Her dedication and innovative approach have 
    significantly contributed to the success of this project.

    My appreciation also extends to the faculty and staff at the Informatics 
    Department. The education and support system provided by the professors have
    been fundamental to my academic and professional growth. Their dedication to
    creating a nurturing and challenging academic environment has been critical 
    to my development.

    Special thanks go to Valéria, my best friend, for her constant support and 
    love. Her encouragement has been a pillar of strength in difficult times, 
    reminding me of the value of perseverance.

    I am grateful to all my friends for their continued encouragement and belief
    in my abilities. Their support and the occasional much-needed distractions 
    have helped me maintain my focus and passion for my research.

    Finally, I would like to thank Marina and Pedro, my psychologist and 
    psychiatrist. Their professional support has been crucial in managing the 
    stress of my academic commitments, significantly contributing to my 
    well-being and success.

    This thesis reflects the collective support and encouragement of everyone 
    mentioned, and I am deeply thankful to each of you.

\begin{abstract}
    \lipsum[1]
\end{abstract}

\begin{translatedabstract}
    \lipsum[1]
\end{translatedabstract}

% \listoffigures

% \listoftables

\begin{listofabbrv}{ABCD} % longest abbreviation
    \item[ABC] Advanced Blockchain Cryptography
    \item[AES] Advanced Encryption Standard
    \item[DLP] Discrete Logarithm Problem
    \item[ECC] Elliptic Curve Cryptography
    \item[EHR] Electronic Health Record
    \item[EMR] Electronic Medical Record
    \item[IoMT] Internet of Medical Things
    \item[IPFS] InterPlanetary File System 
    \item[PoS] Proof of Stake
    \item[PoW] Proof of Work
    \item[RSA] Rivest-Shamir-Adleman4
    \item[SIFF] Secure Identity Federation Framework
\end{listofabbrv}

% idem para a lista de símbolos
% \begin{listofsymbols}{$\alpha\beta\pi\omega$}
%     \item[$\sum{\frac{a}{b}}$] Somatório do produtório
%     \item[$\alpha\beta\pi\omega$] Fator de inconstância do resultado
% \end{listofsymbols}

\tableofcontents

\chapter{Introduction}

\section{Motivation}
    \ToDo{Introduce the context of Laura's paper. \cite{laura2023} and it's use of InterPlanetary File System (IPFS) \cite{benet2013ipfs}}
    
    \Nobre{comentário do orientador}

\section{Organization of the Thesis}
This thesis is organized as follows...


\chapter{Literature Review}
    This section delves into the application of blockchain technology in healthcare, exploring its impact on data security, privacy enhancement, operational efficiency, and additional technological perspectives. Each study discussed is highlighted for its contributions and relevance to advancing blockchain in healthcare.
    
    \Graeff{É necessário colocar o significado de AES e RSA por extenso aqui, visto que são siglas de conhecimento comum na área?}

    \textbf{Data Security:} \citet{Memos2021} propose a layered cloud architecture incorporating AES and RSA encryption to enhance the security of e-health data transmission. \citet{Tian2019} use Secure Identity Federation Framework (SIFF) and Hyperledger Fabric to ensure medical data's integrity, availability, and privacy. \citet{Liu2024} develop a blockchain-based Electronic Medical Record (EMR) sharing scheme with IPFS for secure storage and proxy re-encryption for access control. Additionally, \citet{Esposito2018} investigate the use of blockchain to enhance data security in cloud-based healthcare systems, addressing vulnerabilities inherent in traditional security models.
    
    \textbf{Privacy Enhancement:} \citet{Dwivedi2019} propose a blockchain framework for IoT devices in healthcare that minimizes computational demands while enhancing privacy. \citet{Esfahani2024} introduce a comprehensive privacy scheme using zero-knowledge proofs and ring signatures for IoT-based systems. \citet{Zala2024} enhance e-health data privacy through an attribute-based encryption scheme that improves anonymity. \citet{Li2024} create a regulated medical data trading scheme that ensures privacy and compliance using blockchain and zero-knowledge proofs.
    
    \textbf{Operational Efficiency:} \citet{Vazirani2019} review the effectiveness of blockchain in managing healthcare records, noting improvements in interoperability. \citet{Shen2019} introduce MedChain, a framework that efficiently manages and shares healthcare data. \citet{Bhansali2022} integrate ciphertext-policy attribute-based encryption with federated learning for data management in the IoMT. Furthermore, \citet{XuJie2019} develop 'Healthchain,' a scalable blockchain-based framework for privacy-preserving health data management.
    
    \textbf{Cross-Border Data Sharing and Scalability:} \citet{Rahman2020} describe a blockchain-based platform for secure cross-border data sharing, enhancing trust and security. \citet{Saeed2022} examine blockchain's transformative potential in healthcare, particularly its impact on managing, distributing, and processing medical records.
    
    \textbf{Additional Technological Perspectives:} \citet{Hema2019} assess ECC-based encryption mechanisms for securing healthcare data in cloud environments. \citet{Naz2024} explore defences against quantum computing threats in blockchain through exotic signatures. \citet{Eghmazi2024} focus on securing IoT data using Hyperledger Fabric for scalable solutions. \citet{XuChang2019} discuss secure data sharing in cloud-assisted healthcare systems using cryptographic techniques. Lastly, \citet{Karaca2019} explores mobile cloud computing applications in stroke care, indirectly underscoring the need for integrating secure and efficient technology solutions in healthcare.
    
    This comprehensive review underscores blockchain technology's varied applications and significant potential in enhancing healthcare systems. While the studies discussed provide a robust foundation for advancing blockchain applications in healthcare, they also highlight several research gaps that require further investigation.

    \section{Research Opportunities and Preliminary Goals}
        \Graeff{Não tenho certeza se é melhor colocar essa discussão dos gaps da literatura aqui como sessão1. Aceito sugestões.}

        This section outlines vital research areas in blockchain technology for healthcare that require further exploration. These areas, identified based on the limitations and gaps in current literature, present significant opportunities to advance the field. This thesis aims to address these opportunities through targeted research objectives.

        \subsection{Scalability and Real-World Applicability} 
            A critical gap in current research on blockchain in healthcare is the need for substantial empirical evidence regarding proposed frameworks' scalability and real-world applicability. There is an urgent need to conduct large-scale pilot studies and real-world implementations to substantiate the theoretical advantages of blockchain in this sector. Such research is vital to validate the benefits and to pinpoint potential challenges in diverse healthcare settings. Addressing these gaps is essential for generating concrete data and insights that can guide the scalable deployment of blockchain technologies in practical environments.

        \subsection{Regulatory and Ethical Considerations} 
            As blockchain technology becomes more integrated into healthcare systems, understanding its regulatory and ethical implications is crucial. There is a pressing need to investigate how to design blockchain solutions to comply with healthcare regulations and ethical standards. This exploration should focus on critical areas such as patient privacy, data security, and consent management, encompassing theoretical and practical applications. Ensuring blockchain solutions are legally compliant and ethically sound while maintaining operational flexibility and scalability is fundamental to successful integration into healthcare systems. Addressing these considerations is vital for building trust and accepting blockchain technology in healthcare.

        \subsection{Cybersecurity Challenges} 
            The advancement of blockchain technology in healthcare must continuously adapt to the evolving landscape of cybersecurity threats. Significant opportunities exist to develop sophisticated defence mechanisms specifically designed to address the unique security challenges of blockchain's decentralized nature. Critical areas for further exploration include advancing cryptographic techniques, enhancing consensus algorithms, and implementing robust access control frameworks. These elements protect sensitive health data against unauthorized access and cyber-attacks. Addressing these critical vulnerabilities is imperative to ensuring the security and privacy of healthcare information managed via blockchain. By fortifying these areas, we can significantly enhance the trustworthiness and reliability of blockchain applications in healthcare settings.


\chapter{Methodology}


\chapter{Cryptographic Algorithms Selection}


\chapter{Performance Analysis Results}


\chapter{Security Evaluation}


\chapter{Discussion}


\chapter{Future Work}


\chapter{Conclusion}


% \section{Figuras e tabelas}

% Esta seção faz referência às Figuras~\ref{fig:estrutura},~\ref{fig:ex1} e~\ref{fig:ex2}, a título de exemplo. A primeira figura apresenta a estrutura de uma figura. A \emph{descrição} deve aparecer \textbf{acima} da figura. Abaixo da figura, deve ser indicado a origem da imagem, mesmo se essa for apenas os autores do texto.

% A Figura~\ref{fig:ex1} representa o caso mais comum, onde a figura propriamente dita é importada de um arquivo ( neste exemplo o formato é \texttt{eps} ou \texttt{pdf}. Veja a seção \ref{sec:fig_format}). A Figura~\ref{fig:ex2} exemplifica o uso do environment \texttt{picture}, para desenhar usando o próprio~\LaTeX.

% \begin{figure}[h]
%     \caption{Descrição da Figura deve ir no topo}
%     \begin{center}
%         % Aqui vai um includegraphics , um picture environment ou qualquer
%         % outro comando necessário para incorporar o formato de imagem
%         % utilizado.        
%         \begin{picture}(100,100)
%             \put(0,0){\line(0,1){100}}
%             \put(0,0){\line(1,0){100}}
%             \put(100,100){\line(0,-1){100}}
%             \put(100,100){\line(-1,0){100}}
%             \put(10,50){Uma Imagem}
%         \end{picture}    
%     \end{center}
%     \label{fig:estrutura}
%     \legend{Fonte: Os Autores}
% \end{figure}

% \begin{figure}
%     \caption{Exemplo de figura importada de um arquivo e também exemplo de caption muito grande que ocupa mais de uma linha na Lista~de~Figuras}
%     \begin{center}
%         \includegraphics[width=8em]{images/fig.jpg}
%     \end{center}
%     \legend{Fonte: Os Autores}
%     \label{fig:ex1}
% \end{figure}

% % o `[h]' abaixo é um parâmetro opcional que sugere que o LaTeX coloque a
% % figura exatamente neste ponto do texto. Somente preocupe-se com esse tipo
% % de formatação quando o texto estiver completamente pronto (uma frase a mais
% % pode fazer o LaTeX mudar completamente de idéia sobre onde colocar as
% % figuras e tabelas)
% % \begin{figure}[h]
% \begin{figure}
%     \caption{Exemplo de figura desenhada com o environment \texttt{picture}.}
%     \begin{center}
%         \setlength{\unitlength}{.1em}https://www.overleaf.com/project/62e9d968111a266d82bc89ed
%         \begin{picture}(100,100)
%             \put(20,20){\circle{20}}
%             \put(20,20){\small\makebox(0,0){a}}
%             \put(80,80){\circle{20}}
%             \put(80,80){\small\makebox(0,0){b}}
%             \put(28,28){\vector(1,1){44}}
%         \end{picture}
%     \end{center}
%     \legend{Fonte: Os Autores}
%     \label{fig:ex2}
% \end{figure}

% Tabelas são construídas com praticamente os mesmos comandos. Ver a tabela \ref{tbl:ex1}.

% \begin{table}[h]
%     \caption{Uma tabela de Exemplo}
%     % OBS: não use \begin{center}, pois este aumenta o espaçamento entre a caption/legend e a tabela
%     % Para figuras, a aparência é melhor com o espaçamento extra
%     \centering
%         \begin{tabular}{c|c|p{5cm}}
%           \hline
%           \textit{Col 1}  &   \textit{Col 2}  &   \textit{Col 3} \\
%           \hline
%           \hline
%           Val 1           &   Val 2           & Esta coluna funciona como um parágrafo, tendo uma margem definida em 5cm. Quebras de linha funcionam como em qualquer parágrafo do tex. \\
%           Valor Longo     & Val 2             & Val 3 \\
%           \hline
%         \end{tabular}
%     \legend{Fonte: Os Autores}
%     \label{tbl:ex1}
% \end{table}

% \subsection{Formato de Figuras}
% \label{sec:fig_format}

% O LaTeX permite utilizar vários formatos de figuras, entre eles \emph{eps}, \emph{pdf}, \emph{jpeg} e \emph{png}. Programas de diagramação como Inkscape (e mesmo LibreOffice) permitem gerar arquivos de imagens vetoriais que podem ser utilizados pelo LaTeX sem dificuldade. Pacotes externos permitem utilizar SVG e outros formatos.

% Dia e xfig são programas utilizados por dinossauros para gerar figuras vetoriais. Se possível, evite-os.

% \subsection{Classificação dos etc.}

% O formato do instituto de informática define 5 níveis: capítulo, seção, subseção e outros 2 sem nome.

% \subsubsection{Subsubseção}
% Exemplo de uma subsubseção.

% \paragraph{Parágrafo}
% Exemplo de um parágrafo.

% \section{Sobre as referências bibliográficas}

% A classe \emph{iiufrgs} faz uso do pacote \emph{abnTeX2} com algumas alterações
% feitas por Sandro Rama Fiorini. Culpe ele se algo der errado. Agradeça a ele
% pelo que der certo. As modificações dão uma camada de tinta NATBIB-style,
% já que o abntex2 usa uns comandos de citação feitos para alienígenas de 5 braços
% wtf. Exemplos de citação:

% \begin{itemize}
%     \item \emph{cite}: Unicórnios são verdes \cite{Adams2009Conceptual};
%     \item \emph{citep}:Unicórnios são verdes \citep{Adams2009Conceptual};
%     \item \emph{citet}: Segundo \citet{Adams2009Conceptual}, unicórnios são
%     verdes.
%     \item \emph{citen or citenum}: Segundo \citen{Adams2009Conceptual},
%     unicórnios são verdes.
%     \item \emph{citeauthor e citeyearpar}: Segundo artigos de
%     \citeauthor{Adams2009Conceptual} , unicórnios são verdes
%     \citeyearpar{Adams2009Conceptual}.

% \end{itemize}

% O estilo abnt fornecido antigamente pelo UTUG não é mais recomendado, pois não
% produz saída de acordo com as exigências da biblioteca.

% Recomenda-se o uso de bibtex para gerenciar as referências (veja o arquivo
% biblio.bib).

% \section{Mais uma Seção}

% Agora vamos fazer várias seções para termos valores de 2 dígitos no \contentsname.

% \section{Mais uma Seção}
% \section{Mais uma Seção}
% \section{Mais uma Seção}
% \section{Mais uma Seção}
% \section{Mais uma Seção}
% \section{Mais uma Seção}
% \section{Mais uma Seção}
% \section{Mais uma Seção}
% \section{Mais uma Seção}


% referências
% aqui será usado o environment padrao `thebibliography'; porém, sugere-se
% seriamente o uso de BibTeX e do estilo abnt.bst (veja na página do
% UTUG)
% 
% observe também o estilo meio estranho de alguns labels; isso é
% devido ao uso do pacote `natbib', que permite fazer citações de
% autores, ano, e diversas combinações desses

\bibliographystyle{abntex2-alf}
\bibliography{biblio}

\end{document}
