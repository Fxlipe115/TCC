\documentclass[cic,tc,english]{iiufrgs}

\usepackage[utf8]{inputenc}
\usepackage{graphicx}
\usepackage{times}
% \usepackage{palatino}
% \usepackage{mathptmx}       % p/ usar fonte Adobe Times nas fórmulas

\usepackage[alf,abnt-emphasize=bf]{abntex2cite}

\usepackage{lipsum}

\title{
    Securing Distributed Data with Cryptographic Algorithms: A Case Study on 
    IPFS
}
\translatedtitle{
    Assegurando Dados Distribuídos com Algoritmos Criptográficos: Um Estudo de 
    Caso no IPFS
}

\author{Graeff}{Felipe de Almeida}

\advisor[Prof.~Dr.]{Nobre}{Jéferson Campos}

% \date{maio}{2001}
% \location{Itaquaquecetuba}{SP}

\keyword{cryptographic algorithms}
\keyword{IPFS}
\keyword{decentralized technologies}
\keyword{performance analysis}
\keyword{security evaluation}
\keyword{data storage}
\keyword{data exchange}

\translatedkeyword{algoritmos criptográficos}
\translatedkeyword{IPFS}
\translatedkeyword{tecnologias descentralizadas}
\translatedkeyword{análise de desempenho}
\translatedkeyword{avaliação de segurança}
\translatedkeyword{armazenamento de dados}
\translatedkeyword{intercâmbio de dados}

%\settowidth{\seclen}{1.10~}

\begin{document}

\maketitle

% dedicatoria
\clearpage
\begin{flushright}
    \mbox{}\vfill
    \begin{tabular}{p{0.55\linewidth} p{0.45\linewidth}}
        In loving memory of Professor Raul Fernando Weber, a remarkable educator
        and influential figure in the field of computer security. His passion
        for teaching and dedication to his students left an indelible mark on
        all who had the privilege to learn from him. As I embark on my
        undergraduate thesis in computer security, I humbly dedicate this work
        to the memory of a great professor and a charismatic person who inspired
        generations of students, including myself. His guidance and enthusiasm
        ignited my love for this area, and I am forever grateful for the
        knowledge and inspiration he shared. His legacy lives on in the pursuit
        of knowledge and excellence. Rest in peace, Professor Weber.\\
    \end{tabular}
\end{flushright}

\chapter*{Acknowledgements}
    I would like to express my deepest gratitude to my advisor, Professor 
    Jéferson Campos Nobre, for his invaluable guidance and expertise throughout 
    my research journey. His patience and profound knowledge have not only 
    shaped this work but have profoundly influenced my development in the field.

    I am immensely thankful to Laura, whose foundational research was 
    instrumental for this thesis. Her dedication and innovative approach have 
    significantly contributed to the success of this project.

    My appreciation also extends to the faculty and staff at the Informatics 
    Department. The education and support system provided by the professors have
    been fundamental to my academic and professional growth. Their dedication to
    creating a nurturing and challenging academic environment has been critical 
    to my development.

    Special thanks go to Valéria, my best friend, for her constant support and 
    love. Her encouragement has been a pillar of strength in difficult times, 
    reminding me of the value of perseverance.

    I am grateful to all my friends for their continued encouragement and belief
    in my abilities. Their support and the occasional much-needed distractions 
    have helped me maintain my focus and passion for my research.

    Finally, I would like to thank Marina and Pedro, my psychologist and 
    psychiatrist. Their professional support has been crucial in managing the 
    stress of my academic commitments, significantly contributing to my 
    well-being and success.

    This thesis reflects the collective support and encouragement of everyone 
    mentioned, and I am deeply thankful to each of you.

\begin{abstract}
    \lipsum[1]
\end{abstract}

\begin{translatedabstract}
    \lipsum[1]
\end{translatedabstract}

\listoffigures

\listoftables

\begin{listofabbrv}{ABC} % longest abbreviation
    \item[ABC] Advanced Blockchain Cryptography
    \item[DLP] Discrete Logarithm Problem
    \item[ECC] Elliptic Curve Cryptography
    \item[EHR] Electronic Health Record
    \item[EMR] Electronic Medical Record
    \item[PoS] Proof of Stake
    \item[PoW] Proof of Work
\end{listofabbrv}

% idem para a lista de símbolos
% \begin{listofsymbols}{$\alpha\beta\pi\omega$}
%     \item[$\sum{\frac{a}{b}}$] Somatório do produtório
%     \item[$\alpha\beta\pi\omega$] Fator de inconstância do resultado
% \end{listofsymbols}

\tableofcontents

\chapter{Introduction}

\section{Motivation}
Sausage filler: Introduce the context of Laura's paper. \cite{laura2023} and it's use of InterPlanetary File System (IPFS) \cite{benet2013ipfs}

\section{Scope and Methodology}


\section{Organization of the Thesis}
This thesis is organized as follows...


\chapter{Literature Review}

This chapter explores integrating blockchain technology and cryptographic methods in healthcare data management, emphasizing their impact on enhancing data security and privacy.

\section{Blockchain Technology in Healthcare Data Management}
Blockchain technology significantly enhances healthcare systems' data integrity, privacy, and access control. Esposito et al. \cite{Esposito2018} critically examine blockchain as a solution for healthcare data security within cloud environments, suggesting its pivotal role in overcoming current challenges. Furthermore, Dwivedi et al. \cite{Dwivedi2019} explore blockchain's effectiveness in managing data securely and transparently in the healthcare Internet of Things (IoT) ecosystem.

\section{Cryptographic Techniques for Healthcare Data Security}
The application of advanced cryptographic techniques is crucial for protecting patient information. Memos et al. \cite{Memos2021} propose a layered cloud architecture with advanced encryption techniques for e-health data transmission, highlighting the role of cryptography in safeguarding sensitive health information. Sri Vigna Hema and Kesavan \cite{sri2019} focus on using Elliptic Curve Cryptography (ECC) in cloud environments, underscoring its efficiency and security benefits.

\section{Enhancing Privacy and Security in IoT-based Healthcare Systems}
Securing IoT-generated data in healthcare poses significant challenges. Xu et al. \cite{XuChang2019} introduce a novel searchable encryption scheme to facilitate secure and efficient data access, addressing the need for balancing data utility with privacy concerns in cloud-assisted healthcare systems. Additionally, the study by Karaca et al. \cite{Karaca2019} on mobile cloud computing for stroke healthcare indirectly underscores the importance of secure data management practices in mobile healthcare solutions.

\section{Gaps Identified in Existing Research}
Despite the considerable advancements, research still needs empirical evidence on scalability, performance, and user acceptance in diverse healthcare environments. Saeed et al. \cite{Saeed2022} provide a systematic review of blockchain applications in healthcare, setting a foundation for understanding its potential and limitations.

\section{Informing Future Research}
The reviewed literature underscores blockchain and cryptography's critical roles in tackling healthcare data management challenges. This section lays the groundwork for this dissertation's focus on developing a scalable, user-centric framework leveraging these technologies to enhance privacy and security in healthcare systems.

\section{Blockchain Technology in Healthcare Data Management}
Blockchain's role extends to ensuring data integrity and enabling efficient healthcare services. Esposito et al. \cite{Esposito2018} and Saeed et al. \cite{Saeed2022} discuss blockchain's potential as a solution for healthcare data security and privacy. Additionally, Rahman et al. \cite{Rahman2020} emphasize blockchain's capability for secure and accountable cross-border data sharing, which is crucial for global health data management.

\section{Cryptographic Techniques for Healthcare Data Security}
Cryptographic advancements, including ECC and attribute-based encryption, play a vital role. Sri Vigna Hema and Kesavan \cite{sri2019} demonstrate ECC's security benefits in cloud environments, while Xu et al. \cite{XuJie2019} and Zala et al. \cite{Zala2024} explore attribute-based encryption's potential for privacy protection in e-health.

\section{Enhancing Privacy and Security in IoT-based Healthcare Systems}
IoT-based systems require robust security solutions. Dwivedi et al. \cite{Dwivedi2019} illustrate blockchain's effectiveness in securing IoT-generated data. Eghmazi et al. \cite{Eghmazi2024} further this discussion by integrating blockchain with Hyperledger Fabric to enhance data integrity and privacy in IoT systems.

\section{Challenges and Future Directions}
Liu et al. \cite{Liu2024} and Shen et al. \cite{Shen2019} address scalability and efficiency in electronic medical record (EMR) sharing, emphasizing the need for innovative solutions like proxy re-encryption and P2P networks. Bhansali et al. \cite{Bhansali2022} and Li et al. \cite{Li2024} contribute to the dialogue on federated learning and zero-knowledge proofs for regulated data trading, highlighting 

\section{Highly Relevant for Core Research Focus}

    Blockchain: A Panacea for Healthcare Cloud-Based Data Security and Privacy? \cite{Esposito2018}: Central to understanding blockchain's potential in healthcare data security and privacy within cloud environments.
    A Decentralized Privacy-Preserving Healthcare Blockchain for IoT \cite{Dwivedi2019}: Illustrates blockchain's application in securing IoT-generated healthcare data, addressing privacy and security directly.
    Healthchain: A Blockchain-Based Privacy Preserving Scheme for Large-Scale Health Data \cite{XuJie2019}: Proposes a blockchain-based solution for privacy-preserving healthcare data management, highly relevant for its focus on scalability and privacy.
    A Secure and Efficient Electronic Medical Record Data Sharing Scheme Based on Blockchain and Proxy Re-encryption \cite{Liu2024}: Discusses EMR sharing using blockchain and cryptographic methods, aligning closely with your dissertation’s objectives.
    Blockchain-based End-to-End Privacy-Preserving Scheme for IoT-based Healthcare Systems \cite{maryam2024}: Presents a comprehensive scheme for privacy preservation in IoT-based healthcare, offering direct applicability to enhancing data security.

\section{Moderately Relevant for Methodological Insights and Technological Innovations}

    Achieving Searchable and Privacy-Preserving Data Sharing for Cloud-Assisted E-Healthcare System \cite{XuChang2019}; ECC Based Secure Sharing of Healthcare Data in the Health Cloud Environment \cite{sri2019}: Offer insights into cryptographic techniques for secure data sharing, relevant for discussing data privacy mechanisms.
    Designing an Attribute-Based Encryption Scheme with an Enhanced Anonymity Model for Privacy Protection in E-Health \cite{Zala2024}; MedChain: Efficient Healthcare Data Sharing via Blockchain \cite{Shen2019}: Highlight cryptographic solutions and blockchain’s utility in efficient data sharing, providing context for encryption and blockchain's role in healthcare.
    Cloud-Based Secure Data Storage and Access Control for Internet of Medical Things Using Federated Learning \cite{Bhansali2022}; Enhancing IoT Data Security: Using the Blockchain to Boost Data Integrity and Privacy \cite{Eghmazi2024}: Explore blockchain's integration with IoT and federated learning for data security, relevant for discussions on securing IoT-based healthcare data.

\section{Relevant for Contextual Understanding and Supporting Information}

    Blockchain Technology in Healthcare: A Systematic Review \cite{Saeed2022}; Implementing Blockchains for Efficient Health Care: Systematic Review \cite{Vazirani2019}: Provide a broad overview of blockchain in healthcare, helping to justify research needs and contextualize your study within the current landscape.
    An Enhanced and Secure Cloud Infrastructure for e-Health Data Transmission \cite{Memos2021}; Accountable Cross-Border Data Sharing Using Blockchain Under Relaxed Trust Assumption \cite{Rahman2020}: Offer perspectives on blockchain for secure data transmission and cross-border sharing, underscoring blockchain's versatility in healthcare data management.
    Surveying Quantum-Proof Blockchain Security: The Era of Exotic Signatures \cite{Naz2024}; A Trusted and Regulated Data Trading Scheme Based on Blockchain and Zero-Knowledge Proof \cite{Li2024}: Address future-proofing blockchain against quantum threats and regulated data trading, providing insights into advanced blockchain applications and security considerations.

\section{Indirectly Relevant for Additional Technological Perspectives}

    Mobile Cloud Computing Based Stroke Healthcare System \cite{Karaca2019}: While focused on cloud computing for stroke care, it indirectly highlights the importance of integrating secure and efficient technology solutions in healthcare.

\chapter{Methodology}


\chapter{Cryptographic Algorithms Selection}


\chapter{Performance Analysis Results}


\chapter{Security Evaluation}


\chapter{Discussion and Analysis}


\chapter{Future Work}


\chapter{Conclusion}


% \section{Figuras e tabelas}

% Esta seção faz referência às Figuras~\ref{fig:estrutura},~\ref{fig:ex1} e~\ref{fig:ex2}, a título de exemplo. A primeira figura apresenta a estrutura de uma figura. A \emph{descrição} deve aparecer \textbf{acima} da figura. Abaixo da figura, deve ser indicado a origem da imagem, mesmo se essa for apenas os autores do texto.

% A Figura~\ref{fig:ex1} representa o caso mais comum, onde a figura propriamente dita é importada de um arquivo ( neste exemplo o formato é \texttt{eps} ou \texttt{pdf}. Veja a seção \ref{sec:fig_format}). A Figura~\ref{fig:ex2} exemplifica o uso do environment \texttt{picture}, para desenhar usando o próprio~\LaTeX.

% \begin{figure}[h]
%     \caption{Descrição da Figura deve ir no topo}
%     \begin{center}
%         % Aqui vai um includegraphics , um picture environment ou qualquer
%         % outro comando necessário para incorporar o formato de imagem
%         % utilizado.        
%         \begin{picture}(100,100)
%             \put(0,0){\line(0,1){100}}
%             \put(0,0){\line(1,0){100}}
%             \put(100,100){\line(0,-1){100}}
%             \put(100,100){\line(-1,0){100}}
%             \put(10,50){Uma Imagem}
%         \end{picture}    
%     \end{center}
%     \label{fig:estrutura}
%     \legend{Fonte: Os Autores}
% \end{figure}

% \begin{figure}
%     \caption{Exemplo de figura importada de um arquivo e também exemplo de caption muito grande que ocupa mais de uma linha na Lista~de~Figuras}
%     \begin{center}
%         \includegraphics[width=8em]{images/fig.jpg}
%     \end{center}
%     \legend{Fonte: Os Autores}
%     \label{fig:ex1}
% \end{figure}

% % o `[h]' abaixo é um parâmetro opcional que sugere que o LaTeX coloque a
% % figura exatamente neste ponto do texto. Somente preocupe-se com esse tipo
% % de formatação quando o texto estiver completamente pronto (uma frase a mais
% % pode fazer o LaTeX mudar completamente de idéia sobre onde colocar as
% % figuras e tabelas)
% % \begin{figure}[h]
% \begin{figure}
%     \caption{Exemplo de figura desenhada com o environment \texttt{picture}.}
%     \begin{center}
%         \setlength{\unitlength}{.1em}https://www.overleaf.com/project/62e9d968111a266d82bc89ed
%         \begin{picture}(100,100)
%             \put(20,20){\circle{20}}
%             \put(20,20){\small\makebox(0,0){a}}
%             \put(80,80){\circle{20}}
%             \put(80,80){\small\makebox(0,0){b}}
%             \put(28,28){\vector(1,1){44}}
%         \end{picture}
%     \end{center}
%     \legend{Fonte: Os Autores}
%     \label{fig:ex2}
% \end{figure}

% Tabelas são construídas com praticamente os mesmos comandos. Ver a tabela \ref{tbl:ex1}.

% \begin{table}[h]
%     \caption{Uma tabela de Exemplo}
%     % OBS: não use \begin{center}, pois este aumenta o espaçamento entre a caption/legend e a tabela
%     % Para figuras, a aparência é melhor com o espaçamento extra
%     \centering
%         \begin{tabular}{c|c|p{5cm}}
%           \hline
%           \textit{Col 1}  &   \textit{Col 2}  &   \textit{Col 3} \\
%           \hline
%           \hline
%           Val 1           &   Val 2           & Esta coluna funciona como um parágrafo, tendo uma margem definida em 5cm. Quebras de linha funcionam como em qualquer parágrafo do tex. \\
%           Valor Longo     & Val 2             & Val 3 \\
%           \hline
%         \end{tabular}
%     \legend{Fonte: Os Autores}
%     \label{tbl:ex1}
% \end{table}

% \subsection{Formato de Figuras}
% \label{sec:fig_format}

% O LaTeX permite utilizar vários formatos de figuras, entre eles \emph{eps}, \emph{pdf}, \emph{jpeg} e \emph{png}. Programas de diagramação como Inkscape (e mesmo LibreOffice) permitem gerar arquivos de imagens vetoriais que podem ser utilizados pelo LaTeX sem dificuldade. Pacotes externos permitem utilizar SVG e outros formatos.

% Dia e xfig são programas utilizados por dinossauros para gerar figuras vetoriais. Se possível, evite-os.

% \subsection{Classificação dos etc.}

% O formato do instituto de informática define 5 níveis: capítulo, seção, subseção e outros 2 sem nome.

% \subsubsection{Subsubseção}
% Exemplo de uma subsubseção.

% \paragraph{Parágrafo}
% Exemplo de um parágrafo.

% \section{Sobre as referências bibliográficas}

% A classe \emph{iiufrgs} faz uso do pacote \emph{abnTeX2} com algumas alterações
% feitas por Sandro Rama Fiorini. Culpe ele se algo der errado. Agradeça a ele
% pelo que der certo. As modificações dão uma camada de tinta NATBIB-style,
% já que o abntex2 usa uns comandos de citação feitos para alienígenas de 5 braços
% wtf. Exemplos de citação:

% \begin{itemize}
%     \item \emph{cite}: Unicórnios são verdes \cite{Adams2009Conceptual};
%     \item \emph{citep}:Unicórnios são verdes \citep{Adams2009Conceptual};
%     \item \emph{citet}: Segundo \citet{Adams2009Conceptual}, unicórnios são
%     verdes.
%     \item \emph{citen or citenum}: Segundo \citen{Adams2009Conceptual},
%     unicórnios são verdes.
%     \item \emph{citeauthor e citeyearpar}: Segundo artigos de
%     \citeauthor{Adams2009Conceptual} , unicórnios são verdes
%     \citeyearpar{Adams2009Conceptual}.

% \end{itemize}

% O estilo abnt fornecido antigamente pelo UTUG não é mais recomendado, pois não
% produz saída de acordo com as exigências da biblioteca.

% Recomenda-se o uso de bibtex para gerenciar as referências (veja o arquivo
% biblio.bib).

% \section{Mais uma Seção}

% Agora vamos fazer várias seções para termos valores de 2 dígitos no \contentsname.

% \section{Mais uma Seção}
% \section{Mais uma Seção}
% \section{Mais uma Seção}
% \section{Mais uma Seção}
% \section{Mais uma Seção}
% \section{Mais uma Seção}
% \section{Mais uma Seção}
% \section{Mais uma Seção}
% \section{Mais uma Seção}


% referências
% aqui será usado o environment padrao `thebibliography'; porém, sugere-se
% seriamente o uso de BibTeX e do estilo abnt.bst (veja na página do
% UTUG)
% 
% observe também o estilo meio estranho de alguns labels; isso é
% devido ao uso do pacote `natbib', que permite fazer citações de
% autores, ano, e diversas combinações desses

\bibliographystyle{abntex2-alf}
\bibliography{biblio}

\end{document}
